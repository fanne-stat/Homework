
	\documentclass{article}
	\usepackage{amsmath,amssymb}
	\usepackage[inline]{enumitem}
	\usepackage{blindtext}
	\usepackage{booktabs}
	\usepackage{graphicx}
	\usepackage{xcolor}
	\usepackage[vmargin = 1.5in, top = 1in, bottom = 1.2in, letterpaper]{geometry}
	\usepackage{listings}
	\usepackage{courier}
	\usepackage{bm}
	\lstset{
	basicstyle = \small\tt,
	keywordstyle = \tt\color{blue},
	commentstyle = \it\color[cmyk]{1,0,1,0},
	stringstyle = \tt\color[RGB]{128,0,0},
	%frame = single,
	backgroundcolor = \color[RGB]{245,245,244},
	breaklines,
	extendedchars = false,
	xleftmargin = 2em,
	xrightmargin = 2em,
	aboveskip = 1em,
	tabsize = 4,
	showspaces = false
	}
	\begin{document}
	
	% \newfontfamily\courier{Courier New}

	
	\title{STAT 500 Homework 10}
	\author{Yifan Zhu}
	\maketitle
	
	\begin{enumerate}[leftmargin = 0 em, label = \arabic*., font = \bfseries]
	\item
	\begin{enumerate}
		\item 
		\[SalePrice = \beta_1 LivingArea + \beta_2 Age + \epsilon\]

		$\beta_1$ means with Living Area increasing by 1 square feet, the change (positive or negative) of conditional expectation of SalePrice.

		$\beta_2$ mean with Age increasing by 1 year, he change (positive or negative) of conditional expectation of Sale Price. 

		\item 
		The $R^2$ value is 0.6865. This means that 68.65\% of the variation in Sale Price can be explained by the multiple linear regression model with Living Area and Age of house.

		\item \ 

		\begin{tabular}{llllll}
		\toprule
Source&DF&Sum of Squares&Mean Square&F Value&Pr $>$ F\\
\midrule
Model&2&1.283227E13&6.416137E12&3199.94&$<.0001$\\
Error&2922&5.85885E12&2005082271&&\\
Corrected Total&2924&1.869112E13&&\\
\bottomrule		


		\end{tabular}

		The null and alternative hypotheses are:
		\[H_0 : \beta_1 = \beta_2 = 0,\, H_a : \textrm{at least one $\beta_i \neq 0$}\]
		$F = 3199.94$ with p-value $< 0.0001$. Since the p-value is so small, we will reject the null hypothesis and conclude at least one of the explanatory variables is significant in explaining the response variable Sale Price.

		\item \ 

		\begin{tabular}{llllll}
		\toprule
Variable&DF&Parameter Estimate&Standard Error&t Value&Pr $> |t|$\\
\midrule
LivingArea&1&102.62349&1.74084&58.95&$<.0001$\\
\bottomrule
		\end{tabular}

		The null and alternative hypotheses are:
		\[H_0: \beta_1 = 0,\, H_a: \beta_1 \neq 0\]
		The test statistic $t = 58.95$ with p-value$ < .0001$. Since the p-value is small, we will reject the null hypothesis and conclude the variable Living Area is statistically significant in the model that also includes Age.


		\item \ 

				\begin{tabular}{llllll}
		\toprule
Variable&DF&Parameter Estimate&Standard Error&t Value&Pr $> |t|$\\
\midrule
Age&1&-1072.88363&28.21248&-38.03&$<.0001$\\
\bottomrule

		\end{tabular}

		The null and alternative hypotheses are:
		\[H_0: \beta_2 = 0,\, H_a: \beta_2 \neq 0\]
		The test statistic $t = -38.03$ with p-value$ < .0001$. Since the p-value is small, we will reject the null hypothesis and conclude the variable Age is statistically significant in the model that also includes Living Area.

	\end{enumerate}

	\item 
	\begin{enumerate}
		\item 
		After adding Basement Area and Total Room to the model with Living Area and Age, the sum of squares for error is reduced by 5.85885E12 - 4.360771E12 = 1.498E12.

		\item 
		After adding Basement Area and Total Room to the model with Living Area and Age, the value of $R^2$ increases by 0.7666 - 0.6865 = 0.0801.

		\item 
		The null and alternative hypotheses are:
		\[H_0: \beta_3 = \beta_4 = 0,\, H_a : \textrm{at least one of $\beta_3$ and $\beta_4$ is non-zero} \]

		The test statistic $F = \frac{1.498E12}{1493926329} = 1002.7$ and p-value $< .0001$. Since the p-value is so small, we will reject the null hypothesis and conclude that at least one of Basement Area and Total Room is statistically significant in the model that includes Living Area and Age.
	\end{enumerate}

\item
\begin{enumerate}
	\item 
	After adding interaction term between Living Area and Total Room to the model that includes Living Area, Age, Basement Area, and Total Room, the sums of squares for error is reduced by  4.360771E12 - 4.201687E12 = 0.159E12.

	\item 
	After adding interaction term between Living Area and Total Room to the model that includes Living Area, Age, Basement Area, and Total Room, the value of $R^2$ increases by  0.7751 - 0.7666 = 0.0085.

	\item \ 

	\begin{tabular}{llllll}
	\toprule
 Variable&DF&Parameter Estimate&Standard Error&t Value&Pr $> |t|$\\
 \midrule
Intercept&1&113188&8526.14565&13.28&$<.0001$\\
LivingArea&1&45.00482&5.91264&7.61&$<.0001$\\
Age&1&-794.81037&25.87570&-30.72&$<.0001$\\
BasementArea&1&61.38687&1.95935&31.33&$<.0001$\\
TotalRoom&1&-17415&1355.06909&-12.85&$<.0001$\\
interaction&1&7.52734&0.71614&10.51&$<.0001$\\
\bottomrule
	\end{tabular}

	The t-test statistic is$ t = 10.51$ with p-value $< .0001$. Since the p-value is small, we will reject the null hypothesis and conclude the interaction term is statistically significant in the model that includes Living Area, Age, Basement Area and Total Room.
\end{enumerate}

	


	
	
	


\end{enumerate}

	\end{document}