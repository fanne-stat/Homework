
	\documentclass{article}
	\usepackage{amsmath,amssymb}
	\usepackage[inline]{enumitem}
	\usepackage{blindtext}
	\usepackage{booktabs}
	\usepackage{graphicx}
	\usepackage{xcolor}
	\usepackage[vmargin = 1.5in, top = 1in, bottom = 1.2in, letterpaper]{geometry}
	\usepackage{listings}
	\usepackage{courier}
	\usepackage{multicol}
	\usepackage{multirow}
	\usepackage{bm}
	\lstset{
	basicstyle = \small\tt,
	keywordstyle = \tt\color{blue},
	commentstyle = \it\color[cmyk]{1,0,1,0},
	stringstyle = \tt\color[RGB]{128,0,0},
	%frame = single,
	backgroundcolor = \color[RGB]{245,245,244},
	breaklines,
	extendedchars = false,
	xleftmargin = 2em,
	xrightmargin = 2em,
	aboveskip = 1em,
	tabsize = 4,
	showspaces = false
	}
	\begin{document}
	
	% \newfontfamily\courier{Courier New}

	
	\title{STAT 510 Homework 1}
	\author{Yifan Zhu}
	\maketitle
	
	\begin{enumerate}[leftmargin = 0 em, label = \arabic*., font = \bfseries]
	\item 
	\begin{align*}
	\bm A \bm B & = \begin{bmatrix}
		1 & 5 \\
		4 & -1\\
		0 & 1	
	\end{bmatrix}
	\begin{bmatrix}
		3 & 3 & -2 & 4\\
		5 & -1 & 2 & 3
	\end{bmatrix}\\
	& = \begin{bmatrix}
		1 \\ 4 \\ 0
	\end{bmatrix}
	\begin{bmatrix}
		3 & 3 & -2 & 4
	\end{bmatrix} +
	\begin{bmatrix}
		5 \\ -1 \\ 1
	\end{bmatrix} \begin{bmatrix}
		5 & -1 & 2 & 3
	\end{bmatrix}\\
	& = \begin{bmatrix}
		3 & 3 & -2 & 4\\
		12 & 12 & -8 & 16\\
		0 & 0 & 0& 0
	\end{bmatrix}
		\begin{bmatrix}
			25 & -5 & 10 & 15\\
			-5 & 1 & -2 & -3\\
			5 & -1 & 2 & 3
		\end{bmatrix}\\
		& = \begin{bmatrix}
			28 & -2 & 8 & 19\\
			7 & 13 & -10 & 13\\
			5 & -1 & 2 & 3
		\end{bmatrix}\\
	\end{align*}
	

\item 
\begin{enumerate}
	\item 
	$\bm W = \begin{bmatrix}
			1 & 5 \\
			3 & -1
		\end{bmatrix}$, thus $\bm W^{-1} = \begin{bmatrix}
			-0.0625 &  0.3125\\
			0.1875 & -0.0625
		\end{bmatrix}$. Hence
		\[\bm G = \begin{bmatrix}
			(\bm W^{-1})^{T} & \bm 0
		\end{bmatrix}^T = \begin{bmatrix}
			\bm W^{-1}\\
			\bm 0
		\end{bmatrix} = \begin{bmatrix}
			-0.0625 &  0.3125\\
			0.1875 & -0.0625\\
			0 & 0
		\end{bmatrix}\]

		\item 
		From R we have
		\[\bm G = ginv(\bm A) = 
		\begin{bmatrix}
			 0.01414514  &0.04797048\\
             0.18880689 &-0.05535055\\
			 0.02091021  &0.11439114
		\end{bmatrix}\]

\end{enumerate}

\item 
From the definition, for $u \sim t_m (\delta)$, there exist independent $y \sim N(\delta , 1)$ and $w \sim \chi_m^2$ such that $u = \frac{y}{\sqrt{w/m}}$. 

Hence $u^2 =  \frac{y^2}{w/m} = \frac{y/1}{w/m}$. From the definition of non-central chi-squared distribution, we know that $y^2 \sim \chi_1^2 (\delta^2 / 2)$. Also, $y^2$ and $w$ are independent. Then from the definition of $F$ distribution, we have $u^2 \sim F_{1,m}(\delta^2/2)$.

\item 
Let $\bm z = \begin{bmatrix}
	x\\y
\end{bmatrix}$, then $\bm z \sim N(\bm \mu, \bm I_{2\times 2})$, where $\bm \mu = \begin{bmatrix}
	2\\1
\end{bmatrix}$. Then we have $\bm z^T \bm z \sim \chi_2^2 (\bm \mu^T \bm \mu / 2 = 5/2)$. Thus
\begin{align*}
&P(\textrm{\{the string will need to be
longer than 6 units\}})\\
=& P(\bm z^T \bm z > 36) =    1 - pchisq(36, df = 2, ncp = 5) = 0.00014096
\end{align*}

\item 
\begin{enumerate}
	\item 
	Let $\bm z = \begin{bmatrix}
		z_1 \\ z_2
	\end{bmatrix}$, then $\bm z \sim N(\bm 0, \bm I_{2\times 2})$. Thus $z_1 - z_2  = \begin{bmatrix}
		1 & -1
	\end{bmatrix}\bm z\sim N(0,2)$ and hence $(z_1 - z_2)/\sqrt{2} \sim N(0,1)$. From the definition of $\chi^2$ distribution, we know that 
	\[(z_1 - z_2)^2 / 2 = \left((z_1 - z_2)/\sqrt{2}\right)^2 \sim \chi_1^2 \]

	\item 
	$\begin{bmatrix}
		z_1 + z_2 \\
		z_1 - z_2
	\end{bmatrix} = \begin{bmatrix}
		1 & 1\\
		1 & -1
	\end{bmatrix} \begin{bmatrix}
		z_1 \\ z_2
	\end{bmatrix} \sim N(\bm 0, \begin{bmatrix}
		2 & 0\\
		0 & 2
	\end{bmatrix})$. Hence $z_1 + z_2 $and $z_1 - z_2$ are independent, and $(z_1 + z_2)/\sqrt{2} \perp (z_1 - z_2)^2/2$. From (a) we know $(z_1 - z_2)^2 / 2 \sim \chi^2_1$. And $(z_1 + z_2)/\sqrt{2} \sim N(0,1)$. Then from the definition of $t$ distribution, 
	\[(z_1 + z_2)/|z_1 - z_2| = \frac{(z_1 + z_2)/\sqrt{2}}{\sqrt{(z_1 - z_2)^2/2}} \sim t_1\]
\end{enumerate}

\item 
\begin{enumerate}
	\item 
	$\bar{y} = \frac{1}{n} \bm 1^T \bm y$, thus $\bm y - \bar{y}\bm 1 = \bm y - \frac{1}{n} \bm 1 \bm 1^T \bm y = (\bm I - \frac{1}{n} \bm 1 \bm 1^T)\bm y$. Hence

\begin{align*}
s^2 &= \frac{1}{n-1} \sum_{i = 1}^n (y_i - \bar{y})^2 \\
& = \frac{1}{n-1}((\bm I - \frac{1}{n} \bm 1 \bm 1^T)\bm y)^T (\bm I - \frac{1}{n} \bm 1 \bm 1^T)\bm y\\
& = \frac{1}{n-1} \bm y^T (\bm I - \frac{1}{n}\bm 1 \bm 1^T)^T(\bm I - \frac{1}{n}\bm 1 \bm 1^T) \bm y\\
&  = \frac{1}{n-1} \bm y (\bm I - \frac{2}{n} \bm 1 \bm 1^T + \frac{1}{n^2} \bm 1 \bm 1^T \bm 1 \bm 1^T)\bm y\\
&  =\frac{1}{n-1}\bm y^T (\bm I - \frac{1}{n} \bm 1 \bm 1^T) \bm y\\
& = \bm Y^T \frac{1}{n-1}(\bm I - \frac{1}{n} \bm 1 \bm 1^T) \bm y = \bm y^T \bm B \bm y
\end{align*}

Here $\bm B = \frac{1}{n-1} (\bm I - \frac{1}{n}\bm 1 \bm 1^T)$

\item 
$(n-1)s^2 / \sigma^2 = \bm y^T \frac{1}{\sigma^2} (\bm I - \frac{1}{n} \bm 1 \bm 1^T)\bm y$. Let $\bm A = \frac{1}{\sigma^2} (\bm I - \frac{1}{n} \bm 1 \bm 1^T)$, then $\bm A \bm \Sigma = \bm I - \frac{1}{n} \bm 1 \bm 1^T$. From the proof of (a) we know that $\bm A \bm \Sigma = \bm I - \frac{1}{n} \bm 1 \bm 1^T$ is idempotent. $rank(\bm A) = rank(\bm I - \frac{1}{n} \bm 1 \bm 1^T) = tr(\bm I - \frac{1}{n}\bm 1 \bm 1^T) = tr(\bm I) - \frac{1}{n}tr(\bm 1 \bm 1^T) = n - \frac{1}{n} n = n-1 $. Then we know 
\[(n-1)s^2 / \sigma^2 \sim \chi_{n-1}^2 (\bm \mu^T \bm A \bm \mu / 2)\]
We also have $\bm \mu^T\bm A \bm \mu = \mu^2 \bm 1^T \bm A \bm 1 = \frac{\mu^2}{\sigma^2}\bm 1^T(\bm I - \frac{1}{n} \bm 1 \bm 1^T)\bm 1 = \frac{\mu^2}{\sigma^2} (\bm 1^T \bm 1 - \frac{1}{n} \bm 1^T \bm 1 \bm 1^T \bm 1) = \frac{\mu^2}{\sigma^2} (n - \frac{1}{n} n^2) = 0$, hence
\[(n-1)s^2 / \sigma^2 \sim \chi_{n-1}^2 \]

\end{enumerate}
\item 
\begin{enumerate}
	\item 
	If $\bm A = \bm 0$ then $\bm A^T \bm A = \bm 0$.

	If $\bm A^T \bm A = \bm 0$, then $(\bm A^T \bm A)_{ii}$ in the diagonal line should also be 0.  And for every $i$, we have $(\bm A^T \bm A)_{ii} = \sum_{j = 1}^n a_{ji}^2 = 0 \Rightarrow a_{ji} = 0$ for all $j = 1, 2, \cdots, n$ , where $a_{ji}$ is the $(j,i)$ component of $\bm A$. Hence for every element in $\bm A$, we have $a_{ij} = 0, i,j = 1,2,\cdots, n$.Thus $\bm A = \bm 0$.

	\item 
	$\bm X\bm A = \bm X\bm B \Rightarrow \bm X^T \bm X \bm A = \bm X^T \bm X \bm B$.

	$\bm X^T \bm X \bm A = \bm X^T \bm X \bm B \Rightarrow \bm X^T \bm X(\bm A - \bm B) = \bm 0 \Rightarrow (\bm A - \bm B)^T \bm X^T \bm X \bm (\bm A - \bm B) = \bm 0 \Rightarrow (\bm X \bm (\bm A - \bm B))^T (\bm X(\bm A - \bm B)) = \bm 0$. From (a) we know that $\bm X(\bm A - \bm B) = \bm 0 \Rightarrow \bm X \bm A = \bm X \bm B$.
	\item 
	For any generalized inverse $(\bm X^T \bm X)^{-}$, we have $\bm X^T \bm X (\bm X^T \bm X)^{-} \bm X^T \bm X = \bm X^T \bm X$. Let $(\bm X^T \bm X)^{-} \bm X^T \bm X = \bm A$ and $\bm I = \bm B$, from (b) we have 
	\[\bm X (\bm X^T \bm X)^{-} \bm X^T \bm X = \bm X\]

	\item 
	$\bm G$ is a generalized inverse of $\bm A$, then $\bm A \bm G \bm A = \bm A$. Take transpose on both sides we have $\bm A^T \bm G^T \bm A^T = \bm A^T$. $\bm A$ is symmetric then $\bm A^T = \bm A$, then $\bm A \bm G^T \bm A = \bm A$. Hence $\bm G^T$ is also a generalized inverse of $\bm A$.

	\item 
	Denote $\bm G = (\bm X^T \bm X)^{-}$. As $\bm X^T \bm X$ is symmetric, then $\bm G^T$ is also a generalized inverse of $\bm X^T \bm X$. Hence we have 
	\[\bm X \bm G^T \bm X^T \bm X = \bm X\]
	Take transpose on both sides and we have 
	\[\bm X^T \bm X \bm G \bm X^T = \bm X \Rightarrow \bm X^T \bm X (\bm X^T \bm X)^{-} \bm X^T = \bm X^T\]

	\item 
	From (c):
	\[\bm P_{\bm X} \bm P_{\bm X} = \bm X(\bm X^T \bm X)^- \bm X^T \bm X (\bm X^T \bm X)^- \bm X^T = (\bm X(\bm X^T \bm X)^- \bm X^T \bm X) (\bm X^T \bm X)^- \bm X^T = \bm X (\bm X^T \bm X)^- \bm X^T = \bm P_{\bm X} \]
	From (e):
	\[\bm P_{\bm X} \bm P_{\bm X} = \bm X(\bm X^T \bm X)^- \bm X^T \bm X (\bm X^T \bm X)^- \bm X^T = \bm X(\bm X^T \bm X)^- (\bm X^T \bm X (\bm X^T \bm X)^- \bm X^T) = \bm X (\bm X^T \bm X)^- \bm X^T = \bm P_{\bm X} \]

\item 
From (e) we have $\bm X^T = \bm X^T \bm X \bm G_2 \bm X^T$. Hence we can plug this into $\bm X \bm G_1 \bm X^T$ and we have 
\[\bm X \bm G_1 \bm X^T = \bm X \bm G_1 \bm X^T \bm X \bm G_2 \bm X^T = (\bm X \bm G_1 \bm X^T \bm X) \bm G_2 \bm X^T = \bm X \bm G_2 \bm X^T\]
$\bm X \bm G_1 \bm X^T \bm X = \bm X$ is from (c).


\item 
Denote $(\bm X^T \bm X)^- = \bm G = \bm G_1$ and $\bm G^T = \bm G_2$. From (d) we know that $\bm G_1 $ and $\bm G_2$ are both generalized inverse of $\bm X^T \bm X$, and from (g) we have $\bm X \bm G \bm X^T  =\bm X \bm G_1 \bm X^T = \bm X \bm G_2 \bm X^T = \bm X \bm G^T \bm X^T. $ Hence
\[\bm P_{\bm X}^T = \bm X \bm G^T \bm X^T = \bm X \bm G \bm X^T = \bm P_{\bm X}\]
\end{enumerate}








 	\end{enumerate}


	
	
	
	\end{document}