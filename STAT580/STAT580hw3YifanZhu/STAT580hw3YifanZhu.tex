
	\documentclass{article}
	\usepackage{amsmath,amssymb}
	\usepackage[inline]{enumitem}
	\usepackage{blindtext}
	\usepackage{booktabs}
	\usepackage{graphicx}
	\usepackage{xcolor}
	\usepackage[vmargin = 1.5in, top = 1in, bottom = 1.2in, letterpaper]{geometry}
	\usepackage{listings}
	\usepackage{courier}
	\usepackage{multicol}
	\usepackage{multirow}
	\usepackage{bm}
	\usepackage{algorithm}
	\usepackage{algpseudocode}
	\lstset{
	basicstyle = \small\tt,
	keywordstyle = \ttfamily\bfseries\color{blue},
	commentstyle = \it\color[cmyk]{1,0,1,0},
	stringstyle = \tt\color[RGB]{128,0,0},
	%frame = single,
	backgroundcolor = \color[RGB]{245,245,244},
	breaklines,
	extendedchars = false,
	xleftmargin = 2em,
	xrightmargin = 2em,
	aboveskip = 1em,
	tabsize = 4,
	showspaces = false
	showstringspaces = false
	}
	\begin{document}
	
	% \newfontfamily\courier{Courier New}

	
	\title{STAT 580 Homework 3}
	\author{Yifan Zhu}
	\maketitle
	
	\begin{enumerate}[leftmargin = 0 em, label = \arabic*., font = \bfseries]
	\item 
	\begin{enumerate}
		\item 
		The joint density of $(X,U)$ is 
		\[f(x,u) = f(x)f(u) = g(x)\]
		Then
		\begin{align*}
		P(U\leq r(X)) & = P\left(U\leq \frac{q(X)}{\alpha g(X)}\right)\\
		& = \int_{\mathcal{X}} \int_{0}^{\frac{q(x)}{\alpha g(x)}} f(x,u) \mathrm{d}u \mathrm{d}x\\
		& = \int_{\mathcal{X}} \int_{0}^{\frac{q(x)}{\alpha g(x)}} g(x) \mathrm{d}u \mathrm{d}x\\
		& = \int_{\mathcal{X}} g(x) \frac{q(x)}{\alpha g(x)} \mathrm{d}x\\
		& = \frac{1}{\alpha} \int_{\mathcal{X}} q(x) \mathrm{d}x
		\end{align*}


		\item 
		In the same way, we have 
		\begin{align*}
		P(X \in A, U\leq r(X)) & = P\left(X\in A, U\leq \frac{q(X)}{\alpha g(X)}\right)\\
		& = \int_{A} \int_{0}^{\frac{q(x)}{\alpha g(x)}} f(x,u) \mathrm{d}u \mathrm{d}x\\
		& = \int_{A} \int_{0}^{\frac{q(x)}{\alpha g(x)}} g(x) \mathrm{d}u \mathrm{d}x\\
		& = \int_{A} g(x) \frac{q(x)}{\alpha g(x)} \mathrm{d}x\\
		& = \frac{1}{\alpha} \int_{A} q(x) \mathrm{d}x
		\end{align*}

	\end{enumerate}

	We know that 
	\[P(Y \in A) = P(X\in A|U\leq r(X))\]
	Then
	\[P(Y\in A) = \frac{P(X \in A, U \leq r(X))}{P(U \leq r(X))} = \frac{\frac{1}{\alpha}\int_{A}  q(x) \mathrm{d}x}{\frac{1}{\alpha} \int_{\mathcal{X}} q(x) \mathrm{d}x} = \int_{A} \left(q(x)\left/\int_{\mathcal{X}} q(x) \mathrm{d}x\right.\right)\mathrm{d}x = \int_{A} f(x) \mathrm{d}x\]



	

	\item 
	\begin{enumerate}
		\item 
		\begin{align*}
		\frac{1}{C} &= \int_{0}^{\infty} (2 x ^{\theta - 1} + x^{\theta - 1/2}) \mathrm{e}^{-x} \mathrm{d}x\\
		& = 2 \int_{0}^{\infty} x^{\theta - 1} \mathrm{e}^{-x} \mathrm{d}x + \int_{0}^{\infty} x^{\theta - 1/2} \mathrm{e}^{-x} \mathrm{d}x\\
		& = 2 \Gamma(\theta) + \Gamma(\theta + 1/2) 
		\end{align*}
		Thus the normalizing constant 
		\[C = \frac{1}{2 \Gamma(\theta) + \Gamma(\theta + 1/2)}\]

		\item 
		\begin{align*}
		g(x) & = \frac{1}{2 \Gamma (\theta) + \Gamma(\theta + 1/2)} \left(2 x^{\theta - 1}  \mathrm{e}^{-x} + x^{\theta - 1/2} \mathrm{e}^{-x}\right)\\
		& = \frac{2 \Gamma (\theta)}{ 2 \Gamma (\theta)	+ \Gamma(\theta + 1/2)} \frac{1}{\Gamma(\theta)}x^{\theta - 1} \mathrm{e}^{-x} + \frac{\Gamma(\theta + 1/2)}{2 \Gamma(\theta) + \Gamma(\theta + 1/2)} \frac{1}{\Gamma(\theta + 1/2)}x^{\theta - 1/2} \mathrm{e}^{-x}\\
		& = w_1 g_1 (x) + w_2 g_2(x)
		\end{align*}
		$w_1 + w_2 = 1$ amd $g_1$ is from $\Gamma(\theta)$ and $g_2$ is from $\Gamma(\theta + 1/2)$.
		\item 

		For $X_1 \sim \Gamma(\theta), X_2 \sim \Gamma(\theta + 1/2), U \sim \mathrm{Unif}(0,1)$,
		Let 
		\[Z_= \begin{cases}
			X_1 , U \leq \frac{2 \Gamma(\theta)}{2 \Gamma(\theta) + \Gamma(\theta + 1/2)}\\
			X_2 , U > \frac{2 \Gamma(\theta)}{2 \Gamma(\theta) + \Gamma(\theta + 1/2)}
		\end{cases}\]
		Then 
		\begin{align*}
		P(Z \in A) &= P\left(Z \in A | U \leq \frac{2 \Gamma(\theta)}{2 \Gamma(\theta) + \Gamma(\theta + 1/2)}\right) P\left(U \leq \frac{2 \Gamma(\theta)}{2 \Gamma(\theta) + \Gamma(\theta + 1/2)}\right)\\
		&  + P\left(Z \in A | U > \frac{2 \Gamma(\theta)}{2 \Gamma(\theta) + \Gamma(\theta + 1/2)}\right) P\left(U > \frac{2 \Gamma(\theta)}{2 \Gamma(\theta) + \Gamma(\theta + 1/2)}\right)\\
		& = \frac{2 \Gamma(\theta)}{2 \Gamma(\theta) + \Gamma(\theta + 1/2)}P(X_1 \in A) + \frac{\Gamma(\theta + 1/2)}{2 \Gamma(\theta) + \Gamma(\theta + 1/2)} P(X_2 \in A)\\
		& = w_1 \int_{A} g_1 (x) \mathrm{d}x + w_2 \int_{A} g_2 (x) \mathrm{d}x\\
		& = \int_{A} g(x)\mathrm{d}x
		\end{align*}
	Hence, $Z$ has the desired mixture gamma distribution. Then we can generate it like this
	\begin{algorithm}
		\caption{Procedure to sample from $g(x)$}
		\begin{algorithmic}[1]
			\State generate $U \sim \mathrm{Unif}(0,1)$;
			\If {$U \leq \frac{2\Gamma(\theta)}{2 \Gamma(\theta) + \Gamma(\theta + 1/2)}$} 
			\State {generate $X \sim \Gamma(\theta)$;}
			\Else
			
			{generate $X \sim \Gamma(\theta + 1/2)$;}
			\EndIf
		\end{algorithmic}
	\end{algorithm}

	\item 
	$\frac{q(x)}{\alpha g(x)} \leq 1 \Rightarrow \alpha \geq \frac{q(x)}{g(x)}$. 
	\[\frac{q(x)}{g(x)} = \frac{C\sqrt{x + 4}}{\sqrt{x} + 2}\]
	Thus $\alpha = \sup\{\frac{C\sqrt{x + 4}}{\sqrt{x} + 2} : x > 0\} = C.$ Then
	\[r(x) = \frac{q(x)}{\alpha g(x)} = \frac{\sqrt{x + 4}}{\sqrt{x} + 2}\]

	The rejection procedure is 
	\begin{algorithm}
		\caption{Rejection sampling using $g(x)$ as proposal distribution}
		\begin{algorithmic}[1]
			\State generate $U \sim \mathrm{Unif}(0,1)$ and $X \sim g(x)$ independently;
			\If 
			{$U > r(X) = \frac{\sqrt{X + 4}}{\sqrt{X} + 2}$} \Statex {return to Step 1;}
			\Else \State return $X$;
			\EndIf
		\end{algorithmic}
	\end{algorithm}
	\end{enumerate}
	\newpage
	\item 

	\ 

	\lstinputlisting[language = C]{./Codes/p3.c}
	\newpage
	\item 

	\ 

	\lstinputlisting[language = C]{./Codes/p4.c}
	\newpage
	\item 

	\ 
	
	\lstinputlisting[language = C]{./Codes/p5.c}
 	\end{enumerate}

	
	\end{document}